\subsection{转移速率矩阵与转移概率的计算}
\subsubsection{转移概率的连续性}
\begin{proposition}\label{prop:p122-prop1}
    设 $\{P_t,t\geq 0\}$ 是 $S$ 上的随机半群, 则
    \begin{enumerate}
        \item $\forall t\geq 0,i\in S$, 有 $p_t(i,i)>0$
        \item 若 $\exists s>0$, 使 $p_s(i,i)=1$, 则 $\forall t\geq 0,p_t(i,i)=1$
        \item 若 $\exists s>0$, 使 $p_s(i,j)>0$, 则 $\forall t\geq s,p_t(i,j)>0$
    \end{enumerate}
\end{proposition}

\begin{proof}
\begin{enumerate}
    \item $\limitto{t}{0^+}p_t(i,i)=1\Rightarrow \exists\delta>0,\stt \forall t\in [0,\delta], p_t(i,i)>0$

对 $\forall t\geq 0, \exists s\in [0,\delta),n\in \NN$, 有 $t=s+n\delta$. 故由 C-K 方程, 
\[
\begin{aligned}
    p_t(i,i) &\geq p_{n\delta}(i,i)p_s(i,i)\\
    &\overset{\text{C-K}}{\geq} (p_{\delta}(i,i))^np_s(i,i)>0
\end{aligned}
\]
    \item $p_s(i,i)=1\Rightarrow \forall n\in \NN$, 有 $p_{ns}(i,i)\overset{\text{C-K}}{\geq}(p_s(i,i))^n=1$ ($*1$)
    
    (反证法) 假设 $\exists t>0,\stt p_t(i,i)<1\Rightarrow \sum_{k\neq i}p_t(i,k)>0.  \exists j\neq i,\stt p_t(i,j)>0$ ($*2$).
    
    取 $n$ 使 $ns\geq t$, 则 
    \[
    \begin{aligned}
        p_{ns}(i,i) &=1-\sum_{k\neq i}p_{ns}(i,k)\\
        &\leq 1-p_{ns}(i,j)\\
        &\overset{\text{C-K}}{\leq} 1-p_t(i,j)\underbrace{p_{ns-t}(j,j)}_{>0,\ \text{by}\ (1)}\\
        &< 1\quad \text{by}\ (*2), (1)
    \end{aligned}
    \]
    与 $(*1)$ 矛盾.
    \item $p_s(i,j)>0,p_t(i,j)\geq p_s(i,j)p_{t-s}(j,j)\overset{(1)}{>}0, t\geq s$.
\end{enumerate}
\end{proof}

\begin{theorem}\label{thm:p123-thm1}
    $p_t(i,j)$ 关于 $t\geq 0$ 一致连续, 且对 $t\geq s\geq 0, i,j\in S$.
    \begin{equation}
        |p_t(i,j)-p_s(i,j) |\leq 1-p_{t-s}(i,i)
    \end{equation}
\end{theorem}

\begin{proof}
    \[
    \begin{aligned}
        p_t(i,j)-p_s(i,j) &\xlongequal{\text{C-K}} \sum_{k\in S}p_{t-s}(i,k)p_s(k,j)-p_s(i,j)\\
        &=\sum_{k\neq i}p_{t-s}(i,k) p_s(k,j)-[1-p_{t-s}(i,i)]p_s(i,j)\\
        &=: I_1-I_2
    \end{aligned}
    \]
    $I_1\leq \sum_{k\neq i}p_{t-s}(i,k)=1-p_{t-s}(i,i).\ I_2\leq 1-p_{t-s}(i,i)$. 

    因为 $I_1,I_2$ 非负 $\Rightarrow |\LHS|=|I_1-I_2|=\leq I_1\lor I_2\leq 1-p_{t-s}(i,i)$
\end{proof}

\subsubsection{转移概率的可微性与Kolmogorov方程}

由 Prop \ref{prop:p122-prop1} 和 Thm \ref{thm:p123-thm1} 可证可微性, 但即便如此证明也不是件简单的事, 因此只需要知道下述定理存在即可.

\begin{theorem}\label{thm:p124-thm2}
    $P_t$ 在 $t=0$ 处右导数存在, 具体地,
    \begin{enumerate}
        \item $\forall i\in S$, 下列极限存在.
        \[
        \limitto{t}{0^+}\frac{p_t(i,i)-1}{t}=q(i,i):=-\sup_{t\geq 0}\frac{1-p_t(i,i)}{t}\in [-\infty,0]
        \]
        \item $\forall j\neq i$, 下列极限存在
        \[
            \limitto{t}{0^+}\frac{p_t(i,j)}{t}=q(i,j)\in [0,+\infty)
        \]
        \item $\forall i\in S, \sum_{k\neq i}q(i,k)\leq -q(i,i)$
    \end{enumerate}
\end{theorem}

\begin{definition}[密度矩阵/Q矩阵]
    称 $S$ 上的矩阵 $Q=(q(i,j))_{i,j\in S}$ 为密度矩阵, 若
    \begin{enumerate}
        \item $\forall i\in S,q_i:=-q(i,i)\in [0,+\infty]$
        \item $\forall i,j\in S,q(i,j)\in [0,+\infty)$
        \item $\forall i\in S, \sum_{k\neq i}q(i,k)\leq q_i$
    \end{enumerate}
\end{definition}

\begin{definition}
    由 Thm \ref{thm:p124-thm2} 知, $\{P_t,t\geq 0\}$ 在 $0$ 处的右导数矩阵. $Q:=P_0'=(p_0'(i,j))_{i,j\in S}$ 存在且其为密度矩阵. 称此 $Q$ 为 $\{P_t,t\geq 0\}$ 或 $X$ 的转移速率矩阵/无穷小生成元.
\end{definition}

\begin{definition}
    若 $\forall i,q_i=|q(i,i)|=\sum_{k\neq i}q(i,j)<\infty$ 称密度矩阵 $Q$ 为保守的.
\end{definition}

$\CTMC\to \text{转移半群}\to\text{转移速率矩阵}$. 问: 已知 $Q$, 能否得到 CTMC? 这个问题类比 DTMC 则为: 已知转移矩阵, 能否得到 MC?

\begin{theorem}[Kolmogorov向前/向后方程]
    对具有保守的 $Q$ 的随机半群 $\{P_t,t\geq 0\}$ 有
    \[
    \begin{cases}
        P_t'=QP_t & \text{向后}\\
        P_0=I
    \end{cases}
    \quad
    \begin{cases}
        P_t'=P_t Q & \text{向前}\\
        P_0=I
    \end{cases}
    \]
    注: $AB\neq BA, \text{but}\ QP_t=P_tQ$.
\end{theorem}
\[
\CTMC \overset{+\mu_0}{\iff} \{P_t,t\geq 0\} \overset{P_0'}{\Rightarrow} Q\ (\text{保守的})
\]
反之, 保守的 $Q\Rightarrow \{P_t,t\geq 0\}$

Kolmogorov方程存在唯一性的解, 可以由 $Q$ 构造 CTMC. 侯氏定理.

\begin{theorem}
    在适当的正则性条件下, 有 $Q$ 的向前方程与向后方程存在唯一解, 即 CTMC, $\{P_t,t\geq 0\}+\text{初始分布}\mu_0$, $Q$ 一一对应.
\end{theorem}

\begin{example}[例4.7]
    $\{N(t),t\geq 0\}\sim \PoiP(\lambda)$.
    \[
    p_t(i,j)=\begin{cases}
        e^{-\lambda t}\frac{(\lambda t)^{j-i}}{(j-i)!} & j\geq i\\
        0 & j<i
    \end{cases}
    \]
    \[
    Q=\begin{pmatrix}
        -\lambda & \lambda & 0 & \cdots & \cdots & 0\\
        0 & -\lambda & \lambda & 0 \cdots & 0\\
        \vdots & \vdots & -\lambda & \lambda & \cdots\\
        \vdots & \vdots & \vdots & \cdots & \vdots
    \end{pmatrix}
    \]
    即 $Q(i,i)=-\lambda, Q(i,i+1)=\lambda$.
\end{example}

此前并未限定状态空间有限, 当状态空间有限时:

\begin{corollary}
    设状态空间 $S$ 有限, 则
    \[
    P_t=e^{tQ}=\sum_{n\geq 0}\frac{(tQ)^n}{n!}
    \]
\end{corollary}

\begin{example}[两状态的MC]
    \begin{align*}
        \mathbf{Q}=~
        \bordermatrix{
        &\bf1&\bf2 \cr
        \bf1& -\lambda  & \lambda \cr
        \bf2& \mu & -\mu  \cr
        }~
    \end{align*}
    $\lambda,\mu >0$, 求 $P_t$.
\end{example}

\begin{proof}[解]
    由 Kolmogorov 方程, $P_t'=QP_t$, 即
    \[
    \begin{pmatrix}
        p_t'(1,1) & p_t'(1,2)\\
        p_t'(2,1) & p_t'(2,2)\\
    \end{pmatrix}=
    \begin{pmatrix}
        -\lambda & \lambda\\
        \mu & -\mu
    \end{pmatrix}
    \begin{pmatrix}
        p_t(1,1) & p_t(1,2)\\
        p_t(2,1) & p_t(2,2)
    \end{pmatrix}
    \]
    \[
    \begin{cases}
        p_t'(1,1)=-\lambda p_t(1,1) +\lambda p_t(2,1)\\
        p_t'(2,1)=\mu p_t(1,1)-\mu p_t(2,1)
    \end{cases}
    \]
    \[
    \begin{cases}
        (p_t(1,1)-p_t(2,1))' =-(\lambda+\mu) (p_t(1,1)-p_t(2,1))\\
        p_0(1,1)-p_0(2,1)=1-0=1
    \end{cases}
    \]
    $p_t(1,1)-p_t(2,1)=e^{-(\lambda+\mu)t}$. 代回
    \[
    \begin{cases}
        p_t'(1,1)=-\lambda e^{-(\lambda+\mu)t}\\
        p_0(1,1)=1
    \end{cases},\quad
    \begin{cases}
        p_t'(2,1)=-\mu e^{-(\lambda+\mu)t}\\
        p_0(2,1)=0
    \end{cases}
    \]
    \[
    \begin{aligned}
        p_t(1,1)&=\int_0^t -\lambda e^{-(\lambda +\mu)s}ds+1\\
        &=\frac{\lambda}{\lambda +\mu}e^{-(\lambda+\mu)s}\bigg|_0^t+1=\frac{\lambda}{\lambda+\mu}e^{-(\lambda+\mu)t}+\frac{\mu}{\lambda+\mu}
    \end{aligned}
    \]
    \[
    \begin{aligned}
        p_t(2,1)&=\int_0^t \mu e^{-(\lambda +\mu)s}ds+0\\
        &=\frac{-\mu}{\lambda +\mu}e^{-(\lambda+\mu)s}\bigg|_0^t=\frac{\mu}{\lambda+\mu}-\frac{\mu}{\lambda+\mu}e^{-(\lambda+\mu)t}
    \end{aligned}
    \]
\end{proof}

\subsubsection{轨道的跳跃性质}

探讨 $Q$ 表示了马氏链的什么.

\begin{theorem}\label{thm:p127-thm5}
    设 $S$ 值右连续马氏链 $X:=\{X_t,t\geq 0\}$ 具有保守的转移速率矩阵 $Q=(q(i,j))_{i,j\in S}$. 定义首跳时间 $\eta:=\inf \{t>0|X_t\neq X_0\}$. 其中 $\inf \emp=+\infty$. 那么, 对 $\forall t\geq 0$, 有 $\PP_i(\eta>t)=\PP(\eta>t|X_0=i)=e^{-q_i\cdot t}$, 即在 $\PP_i=\PP(\cdot|X_0=i)$ 下, $\eta\sim \EXP(q_i)$, 其中 $q_i=-q(i,i)$.
\end{theorem}

\begin{corollary}
    若 $q(i,i)=0$, 则 $\PP_i(\eta>t)=1, \forall t\geq 0$. 
\end{corollary}
由 $\{\eta=\infty\}=\cap_{n\geq 1}\{\eta >n\}, \PP_i(\eta=\infty)=1$.
\[
    1=\PP(\eta=\infty)=\PP(X_t=X_0,\forall t>0|X_0=i)=\PP(X_t=i,\forall t\geq 0|X_0=i)
\]
即 $i$ 为吸收态.


\begin{proof}[证明 Thm \ref{thm:p127-thm5}]
(Step 1) 先证明一个数分结论.
\[
\begin{aligned}
    \limitto{s}{0^+}(p_{st}(i,i))^{1/s} 
    &=\limitto{s}{0^+} \exp\left(
        \frac{\ln p_{st}(i,i)}{st}\cdot t
    \right)\\
    &\xlongequal{\ln x\sim x-1(x\to 1)}\exp\left(
        \limitto{s}{0^+}\frac{p_{st}(i,i)-1}{st}\cdot t
    \right)\\
    &=\exp (t q(i,i))=e^{-q_i t}
\end{aligned}
\]
(Step 2) Claim: $\PP(\eta>t|X_0=i)=\PP(X_s=i,\forall s\in [0,t]|X_0=i)$

``$\Rightarrow$'' $t<\eta =\inf\{t>0|X_t\neq i\}, X_0=i. \forall s\in [0,t],X_s=i$

``$\Leftarrow$'' $X_s=i,\forall s\in [0,t]\Rightarrow X_t=i \xrightarrow{\text{右连续}} \exists \delta>0,\forall u\in [t,t+\delta], \stt X_u=i, \eta\geq t+\delta>t$.

(Step 3) 
\[
\begin{aligned}
    \PP_i(\eta>t)&=\PP_i(X_s=i,\forall s\in [0,t])\\
    &\xlongequal{\text{右连续}}\PP_i(\bigcap_{n\geq 1}\{X_{kt/2^n}=i,k=0,1,\cdots,2^n\}) \quad (*)\\
    &=\limit{n}\PP_i(X_{kt/2^n}=i,k=0,1,\cdots,2^n)\\
    &=\limit{n}(p_{t/2^n}(i,i))^{2^n}=e^{-q_i t}
\end{aligned}
\]
$(*)$ 是一个数分结论, 回去证明. $\forall s$ 可找到一列 $n$ 逼近.
\end{proof}
令 $T_0=0$, 归纳定义 $T_n:=\inf \{t>0|X_t\neq X_{T_{n-1}}\},n\geq 1$. $T_1$ 即上面的 $\eta$.

$T_n$: 第$n$次跳跃时间 $(T_1=\eta)$

\begin{lemma}
    设右连续马氏链$X$具有保守的$Q$, 则
    \begin{enumerate}
        \item 在 $[0,\limit{n}T_n)$ 上, $X$的轨道为阶梯函数, 即 $X_t=X_{T_n},t\in[T_n,T_{n+1}]$.
        \item 若 $q_i>0$, 则 $\{X_{T_n},n\geq 1\}\sim\DTMC(\hat{P})$, 其中
        \[
        (\hat{P})_{ij}=\hat{p}_{ij}=\frac{q(i,j)}{q(i)}(1-\delta_{ij})
        \]
        称 $\{X_{T_n},n\geq 1\}$ 为 $X$ 的嵌入链.
    \end{enumerate}
\end{lemma}

(走神)

\subsubsection{过程的构造}
设 $Q=(q(i,j))_{i,j\in S}$ 为保守的密度矩阵, 其中
\[
q_i=-q(i,i)=\sum_{k\neq i}q(i,k)
\]
假定 $q(i,i)>0, \forall i\in S$, 假设没有吸收态. (实际上无需此假设也成立, 但此处为了和教材一致)

定义 $\hat{P}:=(\hat{p}_{i,j})_{i,j\in S}$, 其 $\hat{p}_{ij}=\frac{q(i,j)}{q(i)}(1-\delta_{ij})$, 则 $\hat{P}$ 为随机矩阵, 并称其路径矩阵.

两种看过程的角度:
\begin{enumerate}
    \item $X(t,\cdot)$ r.v. $\forall t\geq 0$
    \item $X(\cdot,\omega)\in C([0,+\infty))$
\end{enumerate}

设
\begin{enumerate}
    \item $\{Y_n,n\geq 0\}\sim \DTMC(\mu_0,\hat{P})$
    \item $\tau_0,\tau_1,\cdots \overset{\text{iid}}{\sim} \EXP(1)$, 注: $\tau_n/q_i\sim \EXP(q_i)$
    \item $\{\tau_k,k\geq 0\}\ind \{Y_n,n\geq 0\}$
\end{enumerate}
下面构造 CTMC.
\begin{enumerate}
    \item $t=n$ 时, 处于状态 $Y_0$, $\eta_1=\tau_0/q(Y_0)\sim \EXP(q(i))$ 在 $\PP(\cdot|Y_0=i)$, 记 $\eta_1\sim \EXP(Y_0)$.
    \item $t=T_1=\eta_1$ 时, 跳到状态 $Y_1$, 在 $Y_1$ 待了 $\eta_2=\tau_1/q(Y_1)\sim\EXP(Y_1)$
    \item $t=T_2=\eta_1+\eta_2$ 时, 跳到状态 $Y_2$, 在 $Y_2$ 待了 $\eta_3=\tau_2/q(Y_2)\sim\EXP(Y_2)$
\end{enumerate}
由此类推, $\eta_n=\tau_{n-1}/q(Y_{n-1}),\forall n\geq 1. T_n=\sum_{k=1}^n\eta_k,T_0=0$.

令 $X_t=Y_n$ (当 $t\in [T_n,T_{n+1}]$), 若
\begin{equation}
    \PP(\limit{n}T_n=+\infty|X_0=Y_0=i)=1
    \label{eq:p129-**}
\end{equation}
称 $X$ 为以 $Q$ 为速率矩阵的跳过程, 称 $Y$ 为 $X$ 的嵌入链, $\{\eta_k,k\geq 1\}$ 为 $X$ 的等待时间序列, $T_n$ 为第 $n$ 次跳的时刻.

注: $\forall t\geq 0$, 至多有限个 $n, \stt T_n\leq t$.

\begin{lemma}
    以 $Q$ 为速率矩阵的跳过程 $X$ 是一个 CTMC, 且 $X\sim \CTMC(Q)$ 以及 $P_t'=QP_t,P_t'=P_t Q$. 其中 $\{P_t,t\geq 0\}$ 是 $X$ 的转移半群.
\end{lemma}

注: $X\sim \CTMC(\mu_0,(P_t)_{t\geq 0})$ (有限维分布族)

$\Rightarrow X\sim \CTMC(\mu_0,Q)$

$\Leftarrow$ Kolmogorov方程的适定性

$\tilde{X}\sim \text{跳过程}(\mu_0,Q)\sim \CTMC (\mu_0,(\tilde{P}_t)_{t\geq 0})$

且 Kolmogorov 方程 $\tilde{P}_t'=\tilde{P}_t Q=Q \tilde{P}_t, \Rightarrow P_t=\tilde{P}_t$, 即 $X\overset{(d)}{=}\tilde{X}$.

\eqref{eq:p129-**} 不好验证, 有什么好验证的充分条件使其成立吗?

\begin{lemma}
    若 $\{q_i,i\in S\}$ 有界, 则条件 \eqref{eq:p129-**} 成立. 特别地, $S$ 有限, 则条件 \eqref{eq:p129-**} 成立.
\end{lemma}

\begin{example}[纯生过程]
    $S=\{1,2,3,\cdots\}, \lambda_1,\lambda_2,\cdots$ 为一列正数. 令 $q_i=q(i,i+1)=\lambda_i,\forall i\geq 1$, 称出生速率.
    \[
    Q=\begin{pmatrix}
        -\lambda_1 & \lambda_1 & 0 & \cdots &\cdots & \cdots\\
        0 & -\lambda_2 & \lambda_2 & 0 &\cdots & \cdots\\
        0 & 0 & -\lambda_3 & \lambda_3 & 0 &\cdots \\
        \vdots & \vdots & & \ddots & \ddots & 
    \end{pmatrix}
    \]
\end{example}

转移速率图:
\[
1\xrightarrow{\lambda_1} 2\xrightarrow{\lambda_2} 3 \xrightarrow{\lambda_3} 4\cdots
\]
Claim: \eqref{eq:p129-**} 成立 $\iff \sum_{n=1}^{\infty}1/\lambda_n=\infty$.
\begin{enumerate}
    \item Poisson过程 $\PoiP(\lambda)$ 为纯生过程
    \item (Durrett Exa 4.5) $q_i=q(i,i+1)=\lambda i^p,\forall i\geq 1$
    \[
    \sum_{n=1}^{\infty}\frac{1}{\lambda_n}
    =\frac{1}{\lambda}\sum_{n=1}^{\infty}\frac{1}{n^p}
    =\begin{cases}
        <\infty & p>1\\
        \infty & p\leq 1
    \end{cases}
    \]
    $p\in [0,1]$ 时, 可定义由 $Q$ 为速率矩阵的跳过程. $p=0$ 时为 Poisson 过程. 特别地, $p=1$ 时称为 Yule 过程.
    \item $\sum_{n=1}^{\infty}1/\lambda_n=\infty$ 时, 嵌入链 $\{Y_n,n\geq 0\}\sim \DTMC(\hat{P})$, 其中
    \[
    (\hat{P})_{i,j}=\hat{p}_{i,j}=\frac{q(i,j)}{q(i)}(1-\delta_{ij})\Rightarrow \hat{p}_{i,i+1}=1
    \]
\end{enumerate}

\begin{example}[生灭过程]
    $S=\{0,1,2,3,\cdots\}$ 令 $q(i,i+1)=\lambda_i,\forall i\geq 0$ (出生速率). $q(i,i-1)=\alpha_i,\forall i\geq 1$ (死亡速率), 其他为0.
    \[
    q_i=-q(i,i)=\sum_{k\neq i}q(i,k)=\begin{cases}
        \lambda_i+\alpha_i & i\geq 1\\
        \lambda_i & i=0
    \end{cases}
    \]
\end{example}
转移速率图:
\[
0
\xleftrightharpoons[\lambda_0]{\alpha_1} 1
\xleftrightharpoons[\lambda_1]{\alpha_2} 2
\xleftrightharpoons[\lambda_2]{\alpha_3} 3
\]
Claim: 若 $\exists c>0$, 使 $\lambda_i\lor \alpha_i\leq c_i(\forall i\geq 1)$, 则 \eqref{eq:p129-**} 成立. 故而此时可定义 $Q=(q(i,j))_{i,j\in S}$ 对应的跳过程, 称其为生灭过程. (证明不是件容易的事, 因此只记结论)

嵌入链 $\{Y_n,n\geq 0\}\sim\DTMC(\hat{P})$, 其中
\[
(\hat{P})_{i,j\in S}=: \hat{p}_{ij}=\frac{q(i,j)}{q(i)}(1-\delta_{ij})=
\begin{cases}
    \lambda_0/\lambda_0=1 & i=0,j=1\\
    \frac{\lambda_i}{\lambda_i+\alpha_i} & j=i+1,i\geq 1\\
    \frac{\alpha_i}{\lambda_i+\alpha_i} & j=i-1,i\geq 1\\
    0 & \text{otherwise}
\end{cases}
\]
转移概率图
\[
0
\xleftrightharpoons[1]{\alpha_1/(\lambda_1+\alpha_1)} 1
\xleftrightharpoons[\lambda_1/(\lambda_1+\alpha_1)]{\alpha_2/(\lambda_2+\alpha_2)} 2
\xleftrightharpoons[\lambda_2/(\lambda_2+\alpha_2)]{\alpha_3/(\lambda_3+\alpha_3)} 3
\xleftrightharpoons[]{}\cdots
\]
$\{Y_n,n\geq 1\}$ 是生灭链.

\begin{example}[排队系统]
    描述 $t$ 时刻系统里的顾客数/队列长度 (=在等待的顾客数+正在被服务的顾客)
    \begin{enumerate}
        \item 要素
        \begin{itemize}
            \item 到达过程/到达速率
            \item 服务时长/服务速率 (指数分布时)
            \item 服务台数/窗口/服务员
        \end{itemize}
        \item 队列容量
        \item 服务规则
        \begin{enumerate}
            \item 先到先得 (FCFS)
            \item 服务时长相互独立
        \end{enumerate}
        \item M/M/S
        \begin{enumerate}
            \item M: Markovian/Memoryless, 到达过程$\sim \PoiP(\lambda)$
            \item M: Memoryless, 服务时长 $\sim\EXP(\alpha)$
            \item S: 服务台数 $\in [1,+\infty)\cap \NN$
        \end{enumerate}
    \end{enumerate}
\end{example}

\begin{enumerate}
    \item M/M/1. 
    
    $S=\{0,1,2,\cdots\}, q(n,n+1)=\lambda (\forall n\geq 0), q(n,n-1)=\alpha (\forall n\geq 1)$

    由前例知, M/M/1 排队系统是生灭过程.
    \begin{enumerate}
        \item $q(i)$
        \[
        q(i)=-q(i,i)=\begin{cases}
            \lambda & i=0\\
            \sum_{k\neq i}q(i,k)=\lambda +\alpha & \forall i\geq 1
        \end{cases}
        \]
        \item 嵌入链 $\{Y_n,n\geq 1\}\sim \DTMC(\hat{P})$, 其中 $\displaystyle (\hat{P})_{i,j}=:\hat{p}_{ij}=\frac{q(i,j)}{q(i)}(1-\delta_{ij})$. 
    \end{enumerate}
    $\displaystyle \Rightarrow \hat{p}_{0,1}=\frac{\lambda}{\lambda}=1,\hat{p}_{i,i+1}=\frac{\lambda}{\lambda+\alpha}(\forall i\geq 1), \hat{p}_{i,i-1}=\frac{\alpha}{\lambda+\alpha}(\forall i\geq 1)$.

    记 $p=\frac{\lambda}{\lambda+\alpha},1-p=\frac{\alpha}{\lambda+\alpha}$, 看出 $\{Y_n,n\geq 1\}$ 是带反射壁的随机游动.
    \item M/M/S
    
    $S=\{0,1,2,\cdots\}, q(n,n+1)=\lambda (\forall n\geq 0)$
    \[
    q(n,n-1)=\begin{cases}
        s\alpha & n\geq s\\
        n\alpha & n\leq s
    \end{cases}
    \]
    由前例知, M/M/S 也是生灭过程
    \item M/M/$\infty$
    
    $S=\{0,1,2,\cdots\}, q(n,n+1)=\lambda (\forall n\geq 0), q(n,n-1)=n\alpha (\forall n\geq 1)$

    由前例知, M/M/$\infty$ 也是生灭过程
\end{enumerate}
\newpage