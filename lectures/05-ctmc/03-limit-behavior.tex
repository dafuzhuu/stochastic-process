\subsection{平稳分布与极限行为}

\subsubsection{平稳分布}

$d(i):=\gcd\{n\geq 1|p_{ii}(n)>0\}$. (正则) CTMC $\Rightarrow t\geq 0,i\in S,p_t(i,i)>0\Rightarrow$ 非周期

\begin{definition}[不可约]
    称 CTMC 不可约, 若 $\forall i,j\in S$, 存在一个状态序列 $k_0=i,k_1,\cdots,k_n=j$, 使 $q(k_{m-1},k_m)>0(1\leq m\leq n)$.
\end{definition}

\begin{lemma}\label{lem:p133-lem1}
    $q(i,j)>0,\forall i\neq j\Rightarrow p_t(i,j)>0,\forall t>0$.
\end{lemma}

\begin{proof}
    \[
    0<q(i,j)=\limitto{h}{0^+}\frac{p_h(i,j)}{h}
    \]
    $\exists \delta>0,\forall s\in (0,\delta],\stt p_s(i,j)>0$. 

    由 Prop \ref{prop:p122-prop1} 知, $p_t(i,j)>0,\forall t>0,i\neq j$.
\end{proof}

\begin{lemma}
    若 $X$ 不可约, 则 $p_t(i,j)>0 (\forall t>0,i,j\in S)$
\end{lemma}

\begin{proof}
    $X$ 不可约 $\Rightarrow$ 对 $i\neq j$, 存在一个状态序列 $k_0=i,k_1,\cdots,k_n=j$, 使 $q(k_{m-1},k_m)>0(1\leq m\leq n)\xrightarrow{\text{Lem}\ \ref{lem:p133-lem1}} p_t(k_{m-1},k_m)>0, (1\leq m\leq n)$
    \[
    p_t(i,j)\overset{\text{C-K}}{\geq } p_{t/n}(k_0,k_1)p_{t/n}(k_1,k_2)\cdots p_{t/n}(k_{n-1},k_n)>0,\forall t>0
    \]
\end{proof}

回忆平稳分布
\begin{itemize}
    \item DTMC: $\pi P=\pi\iff \pi P^n=\pi(\forall n\geq 1)$
    \item CTMC: 没有最小步长 (实数轴不稀疏)
\end{itemize}
因此 CTMC 的平稳分布是这样定义的:

\begin{definition}[平稳分布]
    称概率分布 $\pi$ 为本群 $\{P_t,t\geq 0\}$ 的平稳分布, 若 $\pi P_t=\pi(\forall t\geq 0)$ 
\end{definition}

然而, 这个条件难以验证. 除了转移半群, 还可以用转移速率 $Q$ 刻画.
\[
Q=\begin{cases}
    q(i,j) & i\neq j\\
    q(i,i)=-q(i) & i=j
\end{cases}
\]
其中, $q(i)=\sum_{k\neq i}q(i,k)$ 状态 $i$ 的总转移速率

\begin{lemma}\label{lem:p134-lem4.3}
    $\pi$ 是平稳分布 $\iff \pi Q=0$, 故而 $\pi$ 是 $Q$ 的平稳分布.
\end{lemma}

\begin{proof}
    ``$\Rightarrow$'' $\pi P_t=\pi(\forall t\geq 0)\Rightarrow \frac{d}{dt}(\pi P_t)=0\Rightarrow \pi P_t'=0$. 又 $P_t'=P_t Q$, 故 $\pi P_t Q=0\Rightarrow \pi Q=0$

    ``$\Leftarrow$'' $\pi Q=0\Rightarrow (\pi P_t)'=\pi P_t'=\pi (QP_t)=0$

    $\pi P_t=\pi P_0=\pi$
\end{proof}

\subsubsection{极限行为}

回顾 (DTMC): 不可约, 非周期, 则存在平稳分布$\pi$
\[
\limit{n}P^(n)(i,j)=\pi(j)\quad \forall i,j\in S
\]
\begin{theorem}[Convergence to Equilibrium]
    设 $\{X_t,t\geq 0\}$ 为 CTMC, 转移速率矩阵 $Q$, 不可约. 存在平稳分布 $\pi\Rightarrow \limit{t}p_t(i,j)=\pi(j), \forall i,j\in S$.

    注: 存在 $\pi$ 则唯一 (极限的唯一性)
\end{theorem}

\begin{proof}
    (Step 1) $Q$ 不可约 $\xrightarrow{\text{Lem}\ \ref{lem:p134-lem4.3}} \boxed{p_t(i,j)>0}(\forall t>0,i,j\in S)$. 固定 $h>0$ (步长), 考虑 h-骨架 $\{Z_n:=X_{nh},n\geq 0\}$. 根据定义验证 $Z\sim\DTMC(P_h)$ 且不可约 (方框), 非周期, $\pi$ 也是 $Z$ 的平稳分布.

    $\Rightarrow \limit{n}p_{nh}(i,j)=\pi(j)$. (海涅原理?)

    (Step 2) 
    \[
    \begin{aligned}
        |p_t(i,j)-\pi(j)|&\leq |p_t(i,j)-p_{nh}(i,j)|+|p_{nh}(i,j)-\pi(j)|\\
        &\leq 1-p_{|t-nh|}(i,i)+|p_{nh}(i,j)-\pi(j)|
    \end{aligned}
    \]
    $\forall \epsilon>0,\exists h>0,\stt\forall s\in [0,h], 1-p_s(i,i)<\epsilon/2$

    对同一个 $h, \exists N>0,\stt \forall n>N, |p_{nh}(i,j)-\pi(j)|<\epsilon/2$

    $\forall t\geq Nh, \exists n\geq N, \stt nh\leq t<(n+1)h$, 则 $|t-nh|<h$
    \[
    |p_t(i,j)-\pi(j)|\leq 1-p_{|t-nh|}(i,i)+|p_{nh}(i,j)-\pi(j)|<\epsilon/2+\epsilon/2=\epsilon
    \]
    $\limit{t}p_t(i,j)=\pi(j)$.
\end{proof}

\subsubsection{细致平衡条件(DBC)}

\begin{definition}
    $\pi$为概率分布, 若 $\forall j\neq k$ 有 $\pi(j)q(j,k)=\pi(k)q(k,j)$, 则称 $\pi$ 与 $Q$ 满足DBC.
\end{definition}

\begin{theorem}
    若 $\pi$ 与 $Q$ 满足 DBC $\Rightarrow \pi$ 平稳分布

    注: 反例 4.10
\end{theorem}

\begin{proof}
    DBC $\Rightarrow \pi(k)q(k,j)=\pi(j)q(j,k),\forall k,j\in S$. 关于 $k$ 求和
    \[
    \begin{aligned}
        \sum_{k\in S}\pi(k)q(k,j) &=\pi(j)\sum_{k\in S}q(j,k)\\
        &=\pi(j)\sum_{k\neq j}q(j,k)+\pi(j)q(j,j)\\
        &=0
    \end{aligned}
    \]
    $\therefore \pi Q=0$.
\end{proof}

\begin{example}[生灭链]\label{exa:p136-exa4.12}
    $S=\{0,1,2,\cdots,N\}$或$\{0,1,2,\cdots\}. q(n,n+1)=\lambda_n(\forall n\geq 0), q(n,n-1)=\alpha_n (\forall n\geq 1)$. 设所有 $\lambda_n,\alpha_n>0$ (即不考虑纯生过程), 故不可约. 求满足 DBC 的平稳分布 $\pi$.
\end{example}

\begin{proof}[解]
    DBC $\Rightarrow \pi(n-1)q(n-1,n)=\pi(n)q(n,n-1), \forall n\geq 1$.
    \[
    \pi(n)=\frac{\lambda_{n-1}}{\alpha_n}\pi(n-1)=\frac{\prod_{k=0}^{n-1}\lambda_k}{\prod_{k=1}^n\alpha_k}\pi(0)=:c_n \pi(0), c_0=1
    \]
    \[
    1=\sum_{n=0}^N \pi(n)=\sum_{n=0}^N c_n\pi(0)\iff \pi(0)=\frac{1}{\sum_{n=0}^N c_n}>0
    \]
    记 $c:=\sum_{n=0}^N c_n<\infty$. $\pi(n)=c_n/c(\forall n\geq 0)$ 满足DBC的平稳分布.
\end{proof}

\begin{example}[M/M/$\infty$ 排队系统]
    $S=\{0,1,2,\cdots\},q(n,n+1)=\lambda(\forall n\geq 0), q(n,n-1)=n\alpha(n\geq 1)$.    
\end{example}

令 $c_0=1$,
\[
c_n=\frac{\prod_{k=0}^{n-1}\lambda}{\prod_{k=1}^n(k\alpha)}=\frac{\lambda^n}{n!\alpha^n}(\forall n\geq 1)
\]
\[
c=\sum_{n=1}^{\infty}c_n=\sum_{n=0}^{\infty}\frac{\lambda^n}{n!\alpha^n}=e^{\lambda/\alpha}<\infty
\]
$\pi=(c_0/c, c_1/c, \cdots)$ 为 M/M/$\infty$ 的平稳分布且满足 DBC. $\pi(n)=c_n/c=\frac{\lambda^n}{n!\alpha^n}e^{-\lambda/\alpha}$ 即 $\pi\sim \Poi(\lambda/\alpha)$.

\begin{example}[有止步的M/M/S排队系统]
    $S$个服务员,服务时长$\sim \EXP(\alpha)$. 顾客到达过程 $\sim\PoiP(\lambda)$. $q(n,n+1)=\lambda a_n(\forall n\geq 0)$, 
    \[
    q(n,n-1)=\begin{cases}
        n\alpha & 0\leq n\leq s\\
        s\alpha & n\geq s
    \end{cases}
    \]
\end{example}

\begin{theorem}
    若 $\limit{n}a_n=0$, 则存在一平稳分布
\end{theorem}

\begin{proof}
    令 $c_0:=1, c_n=\prod_{k=0}^{n-1}(\lambda_k)/\prod_{k=0}^n(\alpha_k), \forall n\geq 1$. 若 $c:=\sum_{n=0}^{\infty}c_n<\infty$, 则由例 \ref{exa:p136-exa4.12}, 存在平稳分布 $\pi=(c_0/c,c_1/c,\cdots)$

    下证: $\limit{n}a_n=0\Rightarrow c<\infty$.

    对 $n>s$,
    \[
    c_n=\frac{\prod_{k=0}^{n-1}(\lambda a_k)}{(\prod_{k=0}^s k\alpha)(\prod_{k=s+1}^n s\alpha)}=\prod_{k=0}^{n-1}(\frac{\lambda a_k}{s\alpha})\cdot \frac{\prod_{k=1}^s(s\alpha)}{\prod_{k=1}^s (k\alpha)}=:\prod_{k=0}^{n-1}(\frac{\lambda a_k}{s\alpha})\cdot A
    \]
    $\limit{n}a_n=0 \Rightarrow $ 取 $\epsilon=1/2$, 则 $\exists N>0,\forall n>N,\stt \frac{\lambda a_n}{s\alpha}<1/2$. 对 $\forall n>N\lor s$, 有 $0\leq c_n\leq A\cdot (\frac{1}{2})^n$. $c=\sum_{n=0}^{\infty}c_n<\infty$.
\end{proof}

\subsubsection{访问频率/渐进频率}

回顾 DTMC. 设$X$不可约, $\pi$是平稳分布, 则
\[
\limit{n}\frac{N_n(y)}{n}=\pi(y)=\frac{1}{\EE_y T_y}
\]
其中 $T_y:=\min\{n\geq 1|X_n=y\}, N_n(y)=\sum_{k=0}^n\II_{\{X_k=y\}}$.

对于 CTMC, $\sigma_i:=\inf\{t>\eta|X_t=i\}$, 其中 $\eta:=\inf\{t\geq 0|X_t\neq X_0\}$ 为首次跳跃时刻.

\begin{theorem}[访问频率]
    $X$不可约, $\pi$是平稳分布, 则
    \[
    \limit{t}\frac{\int_0^t \II_{\{X_i=s\}}ds}{t}=\limit{t}\frac{\EE \int_0^t \II_{\{X_i=s\}}ds}{t}=\pi(i)=\frac{1}{\EE_i \sigma_i q_i}
    \]
    注: CTMC下的测度是用积分定义的, $\int_0^t\II_{\{X_i=s\}}ds$ 为 $[0,t]$ 内落到状态$i$的时长.
\end{theorem}

\begin{example}[理发店: 有止步的M/M/1系统]
    $S=\{0,1,2,3\}$, 到达过程 $\sim \PoiP(\lambda),\lambda=2$, 服务时长 $\sim\EXP(\alpha),\alpha=3\text{个/h}$. 均值 $\frac{1}{3}h=20$min. $q(n,n+1)=\lambda(0\leq n\leq 2), q(n,n-1)=\alpha(1\leq n\leq 3)$.
    \begin{enumerate}
        \item[(a)] 求均衡分布.
        \item[(b)] 未接受服务的顾客.
    \end{enumerate}
\end{example}

\begin{proof}[解]
    \begin{enumerate}
        \item[(a)] $S$不可约, 故若存在平稳分布, 则唯一. 由例\ref{exa:p136-exa4.12}知, 有平稳分布 $\pi=(c_0/c,\cdots,c_3/c)$. 其中 $c_0=1,c_n=\prod_{k=0}^{n-1}\lambda/\prod_{k=1}^{n}\alpha=(\frac{\lambda}{\alpha})^n=(\frac{2}{3})^n, 1\leq n\leq 3$.
        \[
        c=1+\frac{2}{3}+(\frac{2}{3})^2+(\frac{2}{3})^3=\frac{1\cdot (1-(2/3)^4)}{1-2/3}=\frac{65}{27}
        \]
        \item[(b)] 
        \[
        \limitto{t}{0}\frac{\int_0^t \II_{\{X_s=3\}}ds}{t}=\pi(3)=(\frac{2}{3})^3/\frac{65}{27}=\frac{8}{65}
        \]
    \end{enumerate}
\end{proof}
\newpage