\subsection{定义}

\begin{definition}[流]
    称一列$\sigma$代数 $\CF_1,\CF_2,\cdots$ 为$\Omega$上的流, 若 $\CF_1\st \CF_2\st \CF_3\st\cdots$
\end{definition}

\begin{definition}[适应过程]
    称一列随机变量 $(X_n)_{n\geq 0}$ 关于流 $(\CF_n)_{n\geq 0}$ 适应的, 若 $\sigma(X_n)\st \CF_n$, 即 $X_n\in \CF_n$, $X_n$关于$\CF_n$可测.
\end{definition}
注: 离散 $\Rightarrow$ 默认 $\sigma$-代数由划分生成, 一系列性质在 Chap 1 中被严格证明.

\begin{example}[自然流]
    设$(X_n)_{n\geq 0}$ 为一列随机变量列, 对$\forall n\geq 0$, 由 Def \ref{def:multi_rv_con_exp}, 令 $\CF_n^X=\sigma(X_0,\cdots,X_n)=(X_0,\cdots,X_n)^{-1}(2^{S_0}\times\cdots\times 2^{S_n})$, 则 $(\CF_n^X)_{n\geq 0}$ 为一个流, 称为 $(X_n)_{n\geq 0}$ 的自然流.
\end{example}

\begin{definition}[离散鞅]\label{def:p140-def3}
    称随机变量列 $(X_n)_{n\geq 0}$ 关于流 $(\CF_n)_{n\geq 0}$ 的鞅, 若
    \begin{enumerate}
        \item (可积性) $\forall n\geq 0, \EE|X_n|<\infty$
        \item (适应性) $(X_n)_{n\geq 0}$ 关于 $(\CF_n)_{n\geq 0}$ 适应
        \item (鞅性) $\forall n\geq 0, \EE(X_{n+1}|\CF_n)\xlongequal{\text{a.s.}}X_n$
    \end{enumerate}
\end{definition}

\begin{theorem}
    \begin{enumerate}
        \item (Durrett, Thm 5.1) 设 $(X_n)_{n\geq 0}$ 关于 $(\CF_n)_{n\geq 0}$ 的鞅, 则 $\EE X_n=\EE X_0$.
        \item 设 $(X_n)_{n\geq 0}$ 为关于 $(\CF_n)_{n\geq 0}$ 适应的可积随机变量列, 则 
        
        $(X_n)_{n\geq 0}$ 为关于 $(\CF_n)_{n\geq 0}$ 的鞅 $\iff$ $\EE(X_{n+1}-X_n|\CF_n)=0,\forall n\geq 0$.
    \end{enumerate}
\end{theorem}

\begin{proof}
    \begin{enumerate}
        \item $\EE X_{n+1}=\EE(\EE(X_{n+1}|\CF_n))\xlongequal{\text{鞅性}}\EE(X_n)\xlongequal{\text{迭代}}\EE X_0$
        \item ``$\Rightarrow$'' $\EE(X_{n+1}-X_n|\CF_n)=X_n-\EE(X_n|\CF_n)\xlongequal{X_n\in \CF_n}X_n-X_n=0$. (提取已知量)
        
        ``$\Leftarrow$'' $\EE(X_{n+1}|\CF_n)=\EE(X_{n+1}-X_n|\CF_n)+\EE(X_n|\CF_n)=X_n$
    \end{enumerate}
\end{proof}

\begin{example}
    常数列 $\{c_n=c,n\geq 0\}$ 关于任意流都是鞅. $c=c\II_{\Omega}, \sigma(c)=\{\emp,\Omega\}\st \CF_n$
\end{example}

\begin{definition}
    称 $(X_n)_{n\geq 0}$ 是关于 $(Y_n)_{n\geq 0}$ 的鞅. 若 $(X_n)_{n\geq 0}$ 关于 $Y$ 的自然域流 $\CF_n^Y=\sigma(Y_0,\cdots,Y_n),n\geq 0$ 是鞅.
\end{definition}

\begin{definition}
    若 Def \ref{def:p140-def3} 中 (3) 为 ``$\leq$'' 时, 即 $\EE(X_{n+1}|\CF_n)\leq X_n$, 称为上鞅. 若为``$\geq$'' 时, 即 $\EE(X_{n+1}|\CF_n)\geq X_n$, 称为下鞅.
\end{definition}

\begin{definition}
    \begin{enumerate}
        \item $X$ 鞅 $\iff$ $X$ 上鞅, 下鞅
        \item $X$ 上鞅 $\iff$ $-X$ 下鞅
    \end{enumerate}
\end{definition}

\begin{theorem}[Durrett, Thm 5.9\& 5.10]
    \begin{enumerate}
        \item 设 $(M_n)_{n\geq 0}$ 关于 $(\CF_n)_{n\geq 0}$ 是一个上鞅, 则 $\EE M_{n+1}\leq \EE M_n$ (期望 $\downarrow$)
        \item 设 $(M_n)_{n\geq 0}$ 关于 $(\mc{S}_n)_{n\geq 0}$ 是一个下鞅, 则 $\EE M_{n+1}\geq \EE M_n$ (期望 $\uparrow$)
    \end{enumerate}
\end{theorem}

\begin{proof}
    (1) $\EE M_{n+1}=\EE(\EE(M_{n+1}|\CF_n))\leq \EE(M_n)$. (2) 同理.
\end{proof}

\begin{theorem}
    设 $(X_n)_{n\geq 0}$ 关于 $(\CF_n)_{n\geq 0}, (\mc{S}_n)_{n\geq 0}$ 均适应的, 且关于 $(\CF_n)_{n\geq 0}$ 是鞅, $\mc{S}_n\st \CF_n (\forall n\geq 0)$, 则
    \begin{enumerate}
        \item $(X_n)_{n\geq 0}$ 关于 $(\mc{S}_n)_{n\geq 0}$ 也是鞅 (小流吃大流)
        \item 特别地, $(X_n)_{n\geq 0}$ 关于其自然域流 $\CF_n^X:=\sigma(X_0,\cdots,X_n), n\geq 0$ 是鞅 
    \end{enumerate}
\end{theorem}

注: $\Pi_1,\Pi_2$ 是 $\Omega$ 上的两个划分, $\Pi_1\st \Pi_2$, 故 $\sigma(\Pi_1)\st \sigma(\Pi_2)$, 则
\[
\EE\bigg(
    \EE(
        \underbrace{X|\sigma(\Pi_1)}_{\in \sigma(\Pi_1)\st \sigma(\Pi_2)}
    )\bigg| \sigma(\Pi_2)
\bigg)=\EE(X|\sigma(\Pi_1))
\]
\[
\EE\bigg(
    \EE(
        X|\sigma(\Pi_1)
    )\bigg| \sigma(\Pi_2)
\bigg)=
\EE\bigg(
    \EE(
        X|\sigma(\Pi_2)
    )\bigg| \sigma(\Pi_1)
\bigg)
\]
\begin{proof}
    (1) (可积性) $X$关于$\CF_n$是鞅 $\Rightarrow$ $\EE|X|<\infty$

    (适应性) $X$ 关于 $\mc{S}$ 适应

    (鞅性) $\EE(X_{n+1}|\mc{S}_n)=\EE(\EE(X_{n+1}|\mc{S}_n|\mc{F}_n)=\EE(\EE(X_{n+1}|\CF_n)|\mc{S}_n)=\EE(X_n|\mc{S}_n)=X_n$
\end{proof}