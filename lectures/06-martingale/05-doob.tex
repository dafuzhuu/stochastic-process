\subsection{Doob极大值不等式与鞅收敛定理}

\begin{theorem}[Doob极大值不等式]
    令 $(X_n)_{n\geq 0}$ 为非负上鞅, $\lambda>0$, 则
    \[
    \PP(\sup_{n\geq 0} X_n\geq \lambda)\leq \frac{\EE X_0}{\lambda}
    \]
    注: (1) $X_n\geq 0 (\forall n\geq 0)$ (2) $(X_n)_{n\geq 0}$ 关于自然流 $(\CF_n^X)_{n\geq 0}$ 为上鞅.
\end{theorem}

若用Markov不等式放缩, 则
\[
\PP(\sup_{n\geq 0} X_n\geq \lambda)\leq \frac{\EE (\sup_{n\geq 0} X_n)}{\lambda}
\]
因为 $\sup$ 是凸函数, 由Jensen不等式, $\EE X_0=\sup_{n\geq 0}\EE(X_0)\leq \EE(\sup_{n\geq 0} X_n)$. 由此可知Doob不等式的结论比Markov不等式更强.

\begin{proof}
    令 $T=\min\{n\geq 0|X_n>\lambda\}$, 则
    \begin{enumerate}
        \item $\{\tau=n\}=\{X_n>\lambda, X_k\leq \lambda,0\leq k\leq n-1\}\in\CF_n^X (\forall n\geq 0)$, 即 $T$ 关于 $(\CF_n^X)_{n\geq 0}$ 为停时.
        \item $\PP(\sup_{n\geq 0}X_n>\lambda)=\PP(\exists n\geq 0, X_n>\lambda)=\PP(T<\infty)=\limit{n}\PP(T\leq n)\ (*)$. 
    \end{enumerate}
    (Step 1) 下证: $\PP(T\leq n)\leq \EE X_0/\lambda \ (\forall n\geq 0)$

    由 Thm \ref{thm:p149-thm5.13} 知, $(X_{\tau\land n})_{n\geq 0}$ 也为上鞅.
    \[
    \EE X_0 \geq \EE X_{\tau\land n}=\EE (X_T\II_{\{T\leq n\}})+\EE(X_n\II_{\{T>n\}})\geq \lambda \PP(T\leq n)
    \]
    $\PP(T\leq n)\leq \EE X_0/\lambda, \forall n\geq 0$.

    (Step 2) 由 (Step 1) 和 $(*)$ 即得 $\PP(\sup_{n\geq 0} X_n\geq \lambda)\leq \EE X_0/\lambda$
\end{proof}

\begin{theorem}[下鞅的Doob极大值不等式]
    设 $(Y_n)_{n\geq 0}$ 为非负下鞅, $\lambda>0$, 则对每一个 $N$, 有
    \[
    \PP(\max_{0\leq k\leq N}Y_k\geq \lambda)\leq \frac{1}{\lambda}\EE\left(
        Y_N\II_{\{\max_{0\leq k\leq N}Y_k\geq \lambda\}}
    \right)\leq \frac{1}{\lambda}\EE (Y_N)
    \]
\end{theorem}

注:
\[
\PP(\max_{0\leq k\leq N}Y_k\geq \lambda)\leq \EE\left(
    \frac{\max_{0\leq k\leq N}Y_N}{\lambda}\II_{\{\max_{0\leq k\leq N}Y_k\geq \lambda\}}=:\zeta
\right)
\]
而 $\lambda^{-1}\EE(Y_N\II_{\{\max_{0\leq k\leq N}Y_k\geq \lambda\}})\leq \zeta$, 说明 Doob不等式是很强的结果.

\begin{proof}
    只需证第一个不等式. 令 $\tau=\min\{n\geq 0|Y_n\geq \lambda\}$, 则
\begin{enumerate}
    \item $\tau$ 关于 $(\CF_n^Y)_{n\geq 0}$ 为停时 (自己写)
    \item $\{\max_{0\leq k\leq N}Y_k\geq \lambda\}=\{\tau\leq N\}=\sum_{k=0}^N\{\tau=k\}$
    \[
    \begin{aligned}
        \lambda \times \RHS &=\EE(Y_N\II_{\{\tau\leq N\}})
        =\sum_{k=0}^N \EE(Y_N\II_{\{\tau=k\}})\\
        &=\sum_{k=0}^N \EE(\EE(Y_N\II_{\{\tau=k\}}|\CF_k^Y))\\
        &=\sum_{k=0}^N \EE(\II_{\{\tau=k\}}\cdot \EE(Y_N|\CF_k^Y))\\
        &\geq \sum_{k=0}^N \EE(\II_{\{\tau=k\}}Y_k)\quad (\text{下鞅是下降的})\\
        &\geq \lambda \sum_{k=0}^N\EE(\II_{\{\tau=k\}})=\lambda \PP(\tau\leq N)=\lambda\times \LHS
    \end{aligned}
    \]
\end{enumerate}
\end{proof}

\begin{theorem}[鞅收敛定理]\label{thm:p160-thm2}
    设 $(X_n)_{n\geq 0}$ 为上鞅 (或下鞅), 且 $\sup_{n\geq 0}\EE |X_n|<\infty$ ($L^1$有界), 则
    \begin{enumerate}
        \item $X_{\infty}:=\limit{n}X_n<\infty$ (a.s.), 即极限存在.
        \item $\EE |X_{\infty}|<\infty$
    \end{enumerate}
\end{theorem}

\begin{corollary}\label{cor:p160-cor}
    设 $(X_n)_{n\geq 0}$ 为非负上鞅, 则
    \begin{enumerate}
        \item $X_{\infty}:=\limit{n}X_n<\infty$ (a.s.)
        \item $\EE |X_{\infty}|<\infty$
        \item $\EE(X_{\infty}|\CF_n^X)\leq X_n (\forall n\geq 0)$ (Fatou's lemma)
    \end{enumerate}
\end{corollary}

\begin{proof}
    $\sup_{n\geq 0}\EE |X_n|=\sup_{n\geq 0}\EE (X_n)\leq \EE X_0$.
\end{proof}

\begin{example}[罐子]
    考虑一个装有红、绿两色的小球.
    \begin{itemize}
        \item 0时刻: 至少一个红、一个绿, 总计 $k\ (k\geq 2)$ 个球 
        \item $n$ 时刻: 抽取一个球, 记住其颜色, 放回, 再放入一个同色球
    \end{itemize}
    故而 \begin{itemize}
        \item 第 $n$ 次抽球时, 罐中有 $k+(n-1)$ 个球
        \item $n$ 时刻动作结束后, 罐中有 $n+k$ 个球
    \end{itemize}
\end{example}

设 $X_n\ (n\geq 1)$ 表示 $n$ 时刻动作结束后, 罐中红球的比例, 则
\[
X_n\in \left\{
    \frac{1}{n+k},\cdots, \frac{n+k-1}{n+k}
\right\}=:S_n,\quad \forall n\geq 0
\]
(1) Claim: $(X_n)_{n\geq 0}$ 为非负鞅, 即 $\EE(X_{n+1}|\CF_n^X)=\EE(X_n),\ \forall n\geq 0$. 适应性, 可积性易证, 只证明鞅性.

\begin{proof}
(Step 1) $\CF_n^X=\sigma(X_0,X_1,\cdots,X_n)=\sigma(\Pi_{(X_0,X_1,\cdots,X_n)})$. 其中,
\[
\Pi_{(X_0,\cdots,X_n)}=\biggl\{
\{(X_0,\cdots,X_n)=(x_0,\cdots,x_n)\}\bigg| (x_0,\cdots,x_n)\in S_0\times\cdots\times S_n    
\biggr\}
\]
因此
\[
\EE(X_{n+1}|\CF_n^X)=\sum_{(x_0,\cdots,x_n)\\\in S_0\times \cdots\times S_n}\EE(X_{n+1}|(X_0,\cdots,X_n)\in (x_0,\cdots,x_n))\II_{\{(X_0,\cdots,X_n)=(x_0,\cdots,x_n)\}}
\]
关于 $\sigma$-代数的条件期望是变量; 关于集合的条件期望是实数, 随机性表现在示性函数上.

(Step 2) 设 $R_n$ 表示 $n$ 时刻动作结束后, 罐中红球的个数, 则
\begin{itemize}
    \item[$1^\circ$] $R_n=(n+k)X_n$
    \item[$2^\circ$] $R_{n+1}=R_n$ 或 $1+R_n$
\end{itemize}
\[
X_{n+1}\xlongequal{1^\circ}\frac{R_{n+1}}{n+k+1}\xlongequal{2^\circ}\frac{R_n}{n+k+1}\II_{\{R_{n+1}=R_n\}}+\frac{1+R_n}{n+k+1}\II_{\{R_{n+1}=1+R_n\}}
\]
\begin{itemize}
    \item[$3^\circ$] 记 $\tilde{\EE}=\EE(\cdot|X_0=x_0,\cdots,X_n=x_n)$
\end{itemize}
\[
\tilde{\EE}(X_{n+1})=\tilde{\EE}\left(
    \frac{R_n}{n+k+1}\II_{\{R_{n+1}=R_n\}}
\right)+\tilde{\EE}\left(
    \frac{1+R_n}{n+k+1}\II_{\{R_{n+1}=1+R_n\}}
\right)
\]
其中 $R_n=(n+k)X_n$, 而 $X_n=x_n$ 为条件.
\[
\begin{aligned}
    \tilde{\EE}(X_{n+1}) &= \frac{(n+k)x_n}{n+k+1}\cdot \tilde{\EE}\II_{\{R_{n+1}=R_n\}}+
    \frac{1+(n+k)x_n}{n+k+1}\tilde{\EE}\II_{\{R_{n+1}=1+R_n\}}\\
    &=\frac{(n+k)x_n}{n+k+1} +\frac{1}{n+k+1}\PP(R_{n+1}=1+R_n|X_0=x_0,\cdots,X_n=x_n)\\
    &=\frac{(n+k)x_n}{n+k+1}+\frac{x_n}{n+k+1}=x_n
\end{aligned}
\tag{*1}
\]
(Step 3) 由 $(*1)$ 有

$\displaystyle\EE(X_{n+1}|\CF_n^X)=\sum_{(x_0,\cdots,x_n)\\\in S_0\times \cdots\times S_n}
x_n\II_{\{(X_0,\cdots,X_n)=(x_0,\cdots,x_n)\}}=X_n\quad (\forall n\geq 0)$
\end{proof}

(2) 由 (1) 知 $X_n(n\geq 0)$ 是非负鞅, 因此是非负上鞅. 故由 Cor \ref{cor:p160-cor} 知, $X_{\infty}:=\limit{n}X_n<\infty$ (a.s.). 问: 当 $k=2$ 时, 求 $X_{\infty}$ 的分布.

(Step 1) 考察 $X_n$ 的分布 ($n$无穷大)
\[
X_n\in \left\{
    \frac{1}{n+k},\cdots,\frac{n+k-1}{n+k}
\right\}\xlongequal{k=2}\left\{
    \frac{1}{n+2},\cdots,\frac{n+1}{n+2}
\right\}
\]
\begin{enumerate}
    \item 每次都抽到绿球
    \[
    \PP(X_n=\frac{1}{n+2})=\frac{1}{2}\times \frac{2}{3}\times\cdots \times\frac{n}{n+1}=\frac{n!}{(n+1)!}
    \]
    \item 有且仅有一次抽到红球
    \[
    \begin{aligned}
        \PP(X_n=\frac{2}{n+2}) &=\sum_{j=1}^n \PP(\text{仅第}j\text{次抽到红球})\\
        &=\sum_{j=1}^n 
        \underbrace{\frac{1}{2}\times \frac{2}{3}\times\cdots\times\frac{j-1}{j}}_{\text{前}j-1\text{次}}\times
        \frac{1}{j+1}\times
        \underbrace{
            \frac{j}{j+2}\times \cdots\times \frac{n-1}{n+1}
        }_{\text{后}n-j\text{次}}\\
        &=\sum_{j=1}^n \frac{1!(n-1)!}{(n+1)!}=\frac{1!\cdot n!}{(n+1)!}
    \end{aligned}
    \]
    \item 有且仅有两次抽到红球
    \[
    \begin{aligned}
        \PP(X_n=\frac{3}{n+2}) &=\sum_{j_1<j_2} \PP(\text{仅第}\ j_1\ \text{次和第}\ j_2\ \text{次抽到红球})\\
        &=\sum_{j_1<j_2} 
        \frac{1}{2}\times \frac{2}{3}\times\cdots\times\frac{j_1-1}{j_1}\times
        \underbrace{\frac{1}{j_1+1}}_{\text{第}j_1\text{次}}\times
        \frac{j_1}{j_1+2}\times \cdots\times \frac{j_2-2}{j_2}\times 
        \underbrace{\frac{2}{j_2+1}}_{\text{第}j_2\text{次}}\times 
        \frac{j_2-1}{j_2+2}\times \cdots \times \frac{n-2}{n+1}\\
        &=\sum_{j_1<j_2} \frac{2!(n-2)!}{(n+1)!}\\
        &=\binom{n}{2} \frac{2!(n-2)!}{(n+1)!}\\
        &=\frac{n!}{2!(n-2)!}\cdot \frac{2!(n-2)!}{(n+1)!}=\frac{1}{n+1}
    \end{aligned}
    \]
    \item 有且仅有 $j-1$ 次抽到红球 ($1\leq j\leq n+1$)
    \[
    \begin{aligned}
        \PP(X_n=\frac{j}{n+2}) &=\PP(\text{前}n\text{次中仅有}j-1\text{次抽到红球})\\
        &=\binom{n}{j-1}\frac{(j-1)!(n-j+1)!}{(n+1)!}\\
        &=\frac{n!}{(j-1)!(n-j+1)!}\cdot \frac{(j-1)!(n-j+1)!}{(n+1)!}=\frac{1}{n+1}
    \end{aligned}
    \]
\end{enumerate}

(Step 2) $0\leq X_n\leq 1\Rightarrow 0\leq X_{\infty}\leq 1$ (a.s.)

$\forall x\in [0,1]$,
\[
\begin{aligned}
    \PP(X_{\infty}\leq x) &=\limit{n}\PP(X_n\leq x)\\
    &=\limit{n}\sum_{j/(n+2)\leq x}\PP(X_n=\frac{j}{n+2})\\
    &=\limit{n}\frac{1}{n+1}\cdot [x(n+2)]=x
\end{aligned}
\]
即 $X_{\infty}\sim U_{[0,1]}$.