成绩:平时(作业+考勤)+期中论文+期末

\section*{概率论准备知识}

概率论中, 随机变量的本质是可测函数. 

\[
X:\Omega\rightarrow S
\]

$S$ 的 $\sigma$-代数记为 $\mathcal{S}$, 是个 Borel $\sigma$-代数(由开集/闭集生成)

Q: 为什么要给 $\Omega$ 一个$\sigma$-代数?

A: 样本空间是抽象的, 给它$\sigma$-代数赋予它结构, 相当于对信息进行重整/提取

概率测度的本质是集函数, 

\[
\text{集合}\rightarrow \text{函数}
\]

将信息具象化, 

\[
\begin{aligned}
    \PP: &\mathcal{F}\rightarrow [0,1]\\
    &A\rightarrow \PP(A)
\end{aligned}
\]

随机过程:一族随机变量 $\{X_t\}_{t\in \mathbb{T}}$

其中 $\mathbb{T}$ 为指标集, $X_t:\Omega\rightarrow S$

\begin{example}
$\mathbb{T}=\mathbb{N}_0$: 时间离散;$\mathbb{T}=[0,T]$: 时间连续 
\end{example}

\[
X:(\Omega, \mathcal{F}, \PP)\rightarrow (S,\mathcal{S}, \mu_X)
\]

思考:什么是随机过程的分布$\{\mu_t\}_{t\in \mathbb{T}}$?

\newpage