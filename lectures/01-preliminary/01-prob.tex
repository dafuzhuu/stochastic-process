\subsection{事件概率}

\subsubsection{事件域}

\begin{definition}[样本空间、事件]
    样本点、样本空间、事件和事件的运算:
    \begin{itemize}
        \item 样本点 $\omega$: 一次试验的结果
        \item 样本空间 $\Omega$: 全体样本点
        \item 事件:$\Omega$ 的子集
        \item 事件的运算:集合的运算, 即交并补($A\cap B, A\cup B, A^c$)
    \end{itemize}
\end{definition}

\begin{definition}
    若 $A\cap B=\varnothing$, 则称 $A$ 与 $B$ 不相交, 更一般地, 若 $A_i\cap A_j=\varnothing (i\neq j)$, 则称 $\{A_i\}_{i\geq 1}$ 互不相交
\end{definition}

\begin{definition}[$\sigma$-代数]
    称 $\mathcal{F}\subset 2^{\Omega}=\{A|A\subset \Omega\}$ 是一个 $\sigma$-代数/事件域(其中 $2^{\Omega}$ 表示所有 $\Omega$ 的子集构成的集合, 是一个集类)

    若\begin{enumerate}
        \item $\Omega\subset \mathcal{F}$
        \item (对补封闭) $A\in \mathcal{F}\rightarrow A^c\in \mathcal{F}$
        \item (对可列并封闭) $A_n\in \mathcal{F}, n\geq 1\Rightarrow \cup_{n\geq 1}A_n=\cup_{n=1}^{\infty} A_n\in\mathcal{F}$
    \end{enumerate}

    $\sigma$代数是满足以上特定条件的集类, 是由 $\Omega$ 的子集构成的集合

    注:$\sigma$代数对有限交/有限并/可列交封闭
\end{definition}

现在给出了一个定义, 我们会想 “为什么定义会这样给呢”, 现在要举一些例子说明 “定义有意义”

\begin{example}
    最小的 $\sigma$代数:$\{\varnothing, \Omega\}$ 

    最大的 $\sigma$代数:$2^{\Omega}$
\end{example}

以上这两个例子一个太小、一个太大, 似乎没意义, 所以叫它们 “平凡的”

\begin{example}
    $A\subset \Omega, \sigma(\{A\})=\sigma(A)=\{A, A^c, \Omega, \varnothing\}=\sigma(A^c)$

    这是由 $A$ 生成的 $\sigma$代数
\end{example}

\begin{definition}[划分/分割]\label{def:partition}
    称 $\Pi_{\Omega}:= \{\Lambda_n, n\geq 1\}$ 是 $\Omega$ 的一个分划, 若 $\Omega=\sum_{n\geq 1}\Lambda_n$

    \begin{enumerate}
        \item $\Omega=\cup_{n\geq 1}\Lambda_n$
        \item $\{\Lambda_n\}_{n\geq 1}$ 互不相交
    \end{enumerate}
\end{definition}

\begin{example}
    $\Omega=\sum_{n\geq 1}\Lambda_n, \Pi_{\Omega}:=\{\Lambda_n\}_{n\geq 1}$

    \[
    \sigma(\Pi_{\Omega})=\left \{\sum_{k\in J}\Lambda_k, J\subset \mathbb{N}\right \}
    \]
\end{example}

\begin{problem}[作业1-1]
    证明:\begin{enumerate}
        \item $\sigma(\Pi_{\Omega})$ 是一个 $\sigma$代数
        \item $\sigma(\Pi_{\Omega})$ 是包含集类 $\Pi_{\Omega}$ 的最小 $\sigma$代数
    \end{enumerate}
\end{problem}


$(S,\mathcal{S})=(S,2^S)$: S 可列时, 取 $2^S$ 为 $\sigma$代数

$(S,\mathcal{S})=(\mathbb{R},\mathcal{B}(\mathbb{R}))$: S 为实数集时, 取博雷尔集 $\mathcal{B}(\mathbb{R})$ 为 $\sigma$代数


\subsubsection{概率测度}

\begin{definition}[概率测度]\label{def:prob_measure}
    $(\Omega, \CF)$ 称 $\PP: \mathcal{F}\rightarrow [0,1]$ 是概率测度
    \begin{enumerate}
        \item 非负性
        \item 归一性
        \item 可列可加性*
    \end{enumerate}
    其中, 可列可加性的表述为:设 $\{E_n, n\geq 1\}$ 是 $\CF$ 中互不相交的集合序列($E_i\cap E_j=\emp, i\neq j$), 则
    \[
    \PP(\cup_{n=1}^{\infty}E_n)=\sum_{n=1}^{\infty}\PP(E_n)
    \]
\end{definition}

\begin{property}
$\PP$ 满足有限可加性(可列可加一定有限可加, 如果既不是可列可加、也不是有限可加, 则不可测)
\end{property}

\begin{corollary}\label{cor:set_operation}
    1. $\PP(A)=1-\PP(A^c)$

    2. 若 $A\subset B$, 则 $\mathbb{B}=\mathbb{A}+\PP(BA^c)\geq \PP(A)$

    3. $\PP(A\cup B)=\PP(A)+\PP(B)-\PP(A\cap B)$
\end{corollary}

\begin{remark}
    引用知乎上\href{https://www.zhihu.com/question/25836213/answer/1204497999}{三维之外}的大白话解释可列可加性:

    首先, 在我们总是习惯于处理有限相加, 而很少遇到无限相加的情况. 从测度论内容理解, 有限相加与事实(数学的)不符, 比如$(0,1)$区间有不可数个点, 每个点的测度(理解为直径吧)是$0$, 按照习惯想法(有限相加), 直径的加和(总宽度)应该为$0$, 显然, $(0,1)$区间的宽度不可能是$0$;
    
    如果规定为“只要是无穷多个点相加, 其宽度就不再是$0$”的话, 还是存在矛盾, 我们知道, 区间$(0,1)$上的有理数是是无穷多个的(而且是可列的), 那么其宽度就应该为$1$, 可是无理数还是不可数的呢——理解为无理数是有理数的无穷大量或有理数是无理数的无穷小量, 那么无理数的宽度是多少呢?即使还是$1$, 显然$(0,1)$区间的宽度不可能是$2$吧!?
    
    于是, 勒贝格说道:在测量长度、面积、体积时, 我们采用可列可加性, 即可列个点相加, 规定其宽度(测度)为$0$, 如果点的个数超过了可列个(这时必是连续统的), 那么, 就不满足了——即这些点的总宽度就不是$0$了 , 而是具有了非$0$的宽度(正测度), 当然, 具有测度的这些点是紧挨在一起的, 否则不一定有测度, 比如康托大师制造的三分集就很诡异. 
    
    到这里, 可列可加性事实上讲完了, 再啰嗦一下次可列可加性. 这是因为不论作为集合, 还是概率上的事件(也是集合), 一般是存在公共元素的, 因此, 一般情形下, 当然满足次可列可加性的性质了, 可列可加性只有在集合之间的距离大于$0$或事件之间完全独立的情形下, 才会满足. 
\end{remark}

\begin{property}[次可列可加性]
    $A_n\subset \mathcal{F}, n\geq 1$

    \[
    \PP(\bigcup_{n\geq 1}A_n)\leq \sum_{n\geq 1}\PP(A_n)
    \]
\end{property}

证明:$\cup_{n\geq 1}A_n=\sum_{n\geq 1}B_n$, 其中 $B_1=A_1, B_2=A_2\cap (A_1)^c,\cdots , B_n=A_n\cap A_1^c\cap A_2^c\cap \cdots \cap A_{n-1}^c$

$B_n\subset A_n$, 由可列可加性和Corollary \ref{cor:set_operation}(2)

\begin{problem}[作业1-2]\label{exer:disjoint_union}
证明 $\cup_{n\geq 1}A_n=\sum_{n\geq 1}B_n$
\end{problem}

证明:
\begin{enumerate}
    \item 先证 \(\bigcup_{n \geqslant 1} A_n \st \sum_{n \geqslant 1} B_n\). 

    假设 \(x \in \bigcup_{n \geqslant 1} A_n\), 

    若 \(x \in A_1\), 则 \(x \in B_1\), 

    若 \(x \in A_2\) 且 \(x \notin A_1\), 则 \(x \in B_2\)

    $\cdots$

    若 \(x \in A_n\) 且 \(x \notin A_1\), \(x \notin A_2\), \(\ldots\), \(x \notin A_{n-1}\), 则 \(x \in B_n\)

    $\forall x\in \bigcup_{n\geq 1}A_n$, 都有 $x\in \bigcup_{n\geq 1}B_n$

    \(\because B_i \cap B_j = \emp,i\neq j\), \(\therefore \bigcup_{n \geqslant 1} B_n = \sum_{n \geqslant 1} B_n\), \(x \in \sum_{n \geqslant 1} B_n\). 

    \item 再证 \(\sum_{n \geqslant 1} B_n \st \bigcup_{n \geqslant 1} A_n\)

    假设 \(x \in \sum_{n \geqslant 1} B_n\), 则 \(\exists n_0 \in \mathbb{N}^+\), 使得 \(x \in B_{n_0}\), 

    由$B$的定义

    \[
    B_{n_0} = A_{n_0} \cap \left( \bigcap_{k=1}^{n_0-1} A_k^c \right)
    \]

    \(\therefore x \in A_{n_0} \subseteq \bigcup_{n \geqslant 1} A_n\)

    \(\therefore \bigcup_{n \geqslant 1} A_n = \sum_{n \geqslant 1} B_n\)\qed
\end{enumerate}

\begin{property}[连续性]\label{prt:measure_continuity}
    (1) $A_n\uparrow$单调上升, 即$A_n\subset A_{n+1}$, $\lim_{n\rightarrow \infty}A_n=\cup_{n\geq 1}A_n$, 则 $\PP(\lim_{n\rightarrow \infty}A_n)=\lim_{n\rightarrow \infty}\PP(A_n)$

    (2) $B_n\downarrow$单调下降, 即$B_n\supset B_{n+1}$, $\lim_{n\rightarrow \infty}B_n=\cap_{n\geq 1}B_n$, 则 $\PP(\lim_{n\rightarrow \infty}B_n)=\lim_{n\rightarrow \infty}\PP(B_n)$
\end{property}

证明:(1) $\cup_{n\geq 1}A_n=A_1+A_2\setminus A_1+A_3\setminus A_2+\cdots$

\[
\begin{aligned}
    \PP(\bigcup_{n\geq 1}A_n)&=\PP(A_1)+\sum_{n\geq 1}\PP(A_{n+1}\setminus A_n)\\
    &=\PP(A_1)+\limit{m}\sum_{n=1}^m \PP(A_{n+1}\setminus A_n)\\
    &=\PP(A_1)+\limit{m}\sum_{n=1}^m [\PP(A_{n+1})-\PP(A_n)]\\
    &=\PP(A_1)+\limit{m}[\PP(A_{m+1})-\PP(A_1)]\\
    &=\limit{m} \PP(A_{m+1})\\
    &=\limit{n} \PP(A_n)\qed
\end{aligned}
\]

(2) $B_n\downarrow B\Rightarrow \forall n, B_{n+1}\st B_n \Rightarrow \forall B_n^c\st B_{n+1}^c$

\[
\begin{aligned}
    \PP(B) = \PP(\cap_{n\geq 1}B_n) &= 1-\PP((\cap_{n\geq 1}B_n)^c)\\
    &=1-\PP(\cup_{n\geq 1}B_n^c)\\
    &=1-\PP(B_1^c\cup (\cup_{n\geq 2}(B_n^c\setminus B_{n-1}^c)))\\
    &=1-\PP(B_1^c)-\sum_{n\geq 2}(\PP(B_n^c)-\PP(B_{n-1}^c))\\
    &=1-\PP(B_1^c)-\limit{m}\sum_{n=2}^m(\PP(B_n^c)-\PP(B_{n-1}^c))\\
    &=1-\PP(B_1^c)-\limit{m}(\PP(B_m^c)-\PP(B_1^c))\\
    &=1-\PP(B_1^c)-\limit{n}\PP(B_n^c)+\PP(B_1^c)\\
    &=1-\limit{n}\PP(B_n^c)\\
    &=\limit{n}\PP(B_n)\qed
\end{aligned}
\]

第二个等式用到De Morgan's Law

\newpage