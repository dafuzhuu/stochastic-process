\subsection{离散时间马氏链}

马尔可夫性 $\leftrightarrow$ 已知现在, 过去与未来不相干/独立

\begin{definition}[(离散时间)马氏链]\label{def:M_1}
    称 $S$ 值随机过程 $\{X_n,n\geq 0\}$ 为马氏链, 若 $X$ 满足以下马氏性:$\forall n\geq 0,x_0,x_1,\cdots,x_n,y\in S$, 
    \begin{equation}
\PP(\underbrace{X_{n+1}=y}_{\text{未来}}|\underbrace{X_0=x_0,\cdots,X_{n-1}=x_{n-1}}_{\text{过去}},\underbrace{X_n=x_n}_{\text{现在}})=\PP(X_{n+1}=y|X_n=x_n)
\label{eq:M1}
\tag{$M_1$}
\end{equation}
    其中 $X_0$ 的分布称为 $X$ 的初始分布
\end{definition}

\begin{definition}
    当 $S$ 为有限集, 称链为有限链, 当 $S$ 为无限集, 称链为无限链
\end{definition}

注:改写 $(M_1)$
\[
LHS=\PP_{X_n=x_n}(X_{n+1}=y|X_0=x_0,\cdots,X_{n-1}=x_{n-1})
\]
\[
RHS=\PP_{X_n=x_n}(X_{n+1}=y)
\]
\[
\begin{aligned}
    M_1 &\Leftrightarrow \{X_{n+1}=y\}\ind_{\{X_n=x_n\}}\{X_0=x_0,\cdots,X_{n-1}=x_{n-1}\}\\
    &\Leftrightarrow X_{n+1}\ind_{\{X_n=x_n\}} (X_0,\cdots,X_{n-1})
\end{aligned}
\]
$(M_1)\quad \text{未来}\ind_{\text{现在}}\text{过去}$
\[
\PP_{\text{现在}}(\text{未来}|\text{过去})=\PP_{\text{现在}}(\text{未来})
\]

\begin{lemma}[马氏性的等价表示]\label{lem:markov_equiv}
    [Grimmett\cite{grimmett}] 下面三个命题等价
    \begin{enumerate}
        \item $(M_1)$ 马氏性
        \item $(M_2)$ $\forall k\geq 0, 0\leq n_1< \cdots<n_k\leq n$, 对于 $y,x_{n_1},\cdots,x_{n_k}\in S$, 
        \begin{equation}
				\PP(X_{n+1}=y|X_{n_1}=x_{n_1},\cdots,X_{n_k}=x_{n_k})=\PP(X_{n+1}=y|X_{n_k}=x_{n_k})
				\label{eq:markov_equiv_M2}
				\end{equation}
        即
        \[
        \{X_{n+1}=y \}\ind_{\{X_{n_k}=x_{n_k}\}}\{X_{n_1}=x_{n_1},\cdots,X_{n_{k-1}}=x_{n_{k-1}}\}
        \]
        \item $(M_3)$ 对 $\forall m\geq 1,n\geq 0, \{y,x_i,0\leq i\leq n\}\st S$, 有
        \begin{equation}
        \PP(X_{n+m}=y|X_0=x_0,\cdots,X_n=x_n)=\PP(X_{n+m}=y|X_n=x_n)
        \label{eq:markov_equiv_M3}
        \end{equation}
        即
        \[
        \{X_{n+m}=y\}\ind_{\{X_n=x_n\}}\{X_0=x_0,\cdots,X_{n-1}=x_{n-1}\}
        \]
    \end{enumerate}
\end{lemma}

证明:思路 $1\leftrightarrows 3\leftrightarrows 2$

$(2)\rightarrow (3)$, 先处理一些记号的问题.记(2)中的 $n$ 为 $n^{(2)}$, (3)中的 $n$ 为 $n^{(3)}$.则取 $n_k=n^{(3)}=n^{(2)}+1-m\leq n^{(2)}$, 所以 $n^{(3)}+m=(n^{(2)}+1-m)+m=n^{(2)}+1$, 即已知(2)可推(3)

$(3)\rightarrow (1)$, 取 $m=1$, 显然

只需证 $(3)\rightarrow (2),(1)\rightarrow (3)$

这里回顾独立的三种写法
\begin{enumerate}
    \item $A\ind_B C$ 记号
    \item $\PP_B(A,C)=\PP_B(A)\PP_B(C)$ 定义
    \item $\PP_B(A|C)=\PP_B(A)$ 定理
\end{enumerate}

(Step 1) 证明 $(3)\rightarrow (2)$

思路:(2)(3)条件不同, 想要由(3)推(2), 则切换到(2)的条件概率测度, 展开, 再用(3)的条件瘦身

对 $\forall k\geq 2, 0\leq n_1<n_2<\cdots<n_k=n$

令 $J=\{0,1,\cdots,n_k-1,n_k\}\backslash \{n_1,\cdots,n_k\}$, $\tilde{\PP}(\cdot)=\PP(\cdot|X_{n_1}=x_{n_1},\cdots,X_{n_k}=x_{n_k})$

\[
\begin{aligned}
    \tilde{\PP}(X_{n+1}=y)&=\sum_{x_j\in S,j\in J}\tilde{\PP}(X_{n+1}=y|X_j=x_j,j\in J)\cdot\tilde{\PP}(X_j=x_j,j\in J)\qquad [\text{全概公式}]\\
    &=\PP(X_{n+1}=y|X_{n_k}=x_{n_k})\sum_{x_j\in S,j\in J}\tilde{\PP}(X_j=x_j,j\in J)\qquad [(3), \PP_C(\cdot|A)=\PP_C(\cdot)]\\
    &=\PP(X_{n+1}=y|X_{n_k}=x_{n_k})
\end{aligned}
\]

其中, 记号 $\sum_{x_j\in S,j\in J}$ 中的下标意为:假设 $J$ 中元素个数为 $\# J=u$, 则 $(x^{(1)},\cdots, x^{(u)})\in S^u$.从简单的开始, $\sum_{x\in S}\PP(X=x)=\PP(\Omega), \sum_{(x,y)\in S^2}\PP(X=x,Y=y)=\PP(\Omega),\cdots$, $\sum_{(x^{(1)},\cdots, x^{(u)})\in S^u}\PP(X^{(1)}=x^{(1)},\cdots,X^{(u)}=x^{(u)})=\PP(\Omega)=1$

(Step 2) 下证 $(1)\rightarrow (3)$

1. $m=1$ 时, 即 $(1)$

2. 假设 $m=k$ 时 $(3)$ 成立, 即 $\forall n\geq 1, \{y,x_i,n\geq i\geq 0\}\st S$,
\[
\{X_{n+k}=y\}\ind_{\{X_n=x_n\}}\{X_0=x_0,\cdots, X_{n-1}=x_{n-1}\}\xrightarrow{\text{性质\eqref{prop:pairwise_indep}}}\{X_{n+k}=y\}\ind_{\{X_n=x_n\}}\{X_{n-1}=x_{n-1}\}
\]
\[
\begin{aligned}
    \PP(X_{n+k}=y|X_0=x_0,\cdots,X_n=x_n)&=\PP(X_{n+k}=y|X_n=x_n)\\
    &=\PP(X_{n+k}=y|X_n=x_n,X_{n-1}=x_{n-1})
\end{aligned}\tag{*}
\]
当 $m=k+1$ 时, 对 $\forall \{y,x_i,n\geq i\geq 0\}\st S$

令 $\tilde{\PP}_n(\cdot):=\PP(\cdot|X_0=x_0,\cdots,X_n=x_n)$
\[
\begin{aligned}
    \tilde{\PP}_n(X_{n+k+1}=y)&\overset{\text{Thm}\eqref{thm:law_total_prob}}{=}\sum_{x_{n+1}\in S}\tilde{\PP}_n(X_{n+k+1}=y|X_{n+1}=x_{n+1})\cdot \tilde{\PP}_n(X_{n+1}=x_{n+1})\\
    &=\sum_{x_{n+1}\in S}\PP(X_{n+k+1}=y|X_{n+1}=x_{n+1},X_n=x_n)\cdot \PP(X_{n+1}=x_{n+1}|X_n=x_n)\quad [\text{(*), 归纳法假设}]\\
    &\overset{\eqref{eq:multiply_func}}{=}\sum_{x_{n+1}\in S}\PP(X_{n+k+1}=y,X_{n+1}=x_{n+1},X_n=x_n)/\PP(X_n=x_n)\\
    &=\PP(X_{n+k+1}=y,X_n=x_n)/\PP(X_n=x_n)\\
    &=\PP(X_{n+k+1}=y|X_n=x_n)
\end{aligned}
\]
即 $m=k+1$ 得证\qed

证明 (Step 2) 时如果在 $x_{n+k}$ 处展开而不是在 $x_{n+1}$, 也是可以的.实际上在 $x_{n+j}, \forall j, 1\leq j\leq k$ 展开都可以, 关键在于用性质\ref{prop:pairwise_indep}和全概公式\ref{thm:law_total_prob}凑出乘法公式\eqref{eq:multiply_func}, 消元即可.

\begin{remark}
    三种写法的直觉
    \begin{enumerate}
        \item $M_1$:未来“下一步”跟过去“每一步”都无关
        \item $M_2$:未来“下一步”跟过去的“任意若干步”都无关
        \item $M_3$:未来“下m步”跟过去“每一步”都无关
    \end{enumerate}
    可以推出, 由(2)(3), 下式也成立:
    
    对 $\forall m\geq 1,n\geq 0, \{y,x_i,0\leq i\leq n\}\st S$
    \[
    \PP(X_{n+m}=y|X_{n_1}=x_{n_1},\cdots,X_{n_k}=x_{n_k})=\PP(X_{n+m}=y|X_{n_k}=x_{n_k})
    \]
\end{remark}

\begin{corollary}\label{cor:markov_con_cut}
    若 $X$ 是马氏链, 则 $\forall n\geq 1,\{x_i,n\geq i\geq 0,y\}\st S$, 有 
    \[
    \PP(X_{n+1}=y|X_0=x_0,\cdots,X_n=x_n)=\PP(X_{n+1}=y|X_n=x_n,X_{n-1}=x_{n-1})
    \]
\end{corollary}

补充记号:
\begin{itemize}
    \item 乘积空间
    \[
        S^n:=\underbrace{S\times\cdots\times S}_{\text{n个}}
    \]
    \item 乘积 $\sigma$ 代数
    \[
        \bigotimes_n 2^S:=\underbrace{2^S\times\cdots\times 2^S}_{\text{n个}}
    \]
\end{itemize}

\begin{property}[马氏性的等价条件]\label{prt:markov_equiv_condition}
下列三个命题等价
\begin{enumerate}
    \item 马氏性 $(M_1)$
    \item 对 $\forall n\geq 1,m\geq 1,A\in \otimes_n 2^S,B\in \otimes_m 2^S$, 即 $(A\st S^n,B\st S^m)$, 有
    \begin{equation}
    \begin{aligned}
        &\PN((X_0,\cdots,X_{n-1})\in A,(X_{n+1},\cdots,X_{n+m})\in B)\\
        =&\PN((X_0,\cdots,X_{n-1})\in A)\cdot \PN((X_{n+1},\cdots,X_{n+m})\in B)
    \end{aligned}
    \label{eq:markov_equiv_condition_M2}
    \end{equation}
    即 $(X_0,\cdots,X_{n-1})\ind_{\{X_n=x_n\}}(X_{n+1},\cdots,X_{n+m})$ 的定义
    \item $\PN((X_{n+1},\cdots,X_{n+m})\in B|(X_0,\cdots,X_{n-1})\in A)=\PN((X_{n+1},\cdots,X_{n+m})\in B)$
\end{enumerate}
\end{property}

证明:$(2)\Leftrightarrow (3)$, 独立的定义和定理, 显然

$(3)\rightarrow (1)$, 取 $k=0$ 显然

只需证 $(1)\rightarrow (3)$

只需证 $(3)$ 对简单事件 $A,B$ (单点集合) 成立, 即 $\forall n\geq 1,m\geq 1, \{x_0,x_1,\cdots,x_{n+m}\st S\}$, 有
\[
\PN((X_{n+1},\cdots,X_{n+m})=x_{n+1}^{n+m}|(X_0,\cdots,X_{n-1})=x_0^{n-1})=\PN((X_{n+1},\cdots,X_{n+m})=x_{n+1}^{n+m})
\]
其中 $x_{n+1}^{n+m}=(x_{n+1},\cdots,x_{n+m}),x_{0}^{n-1}=(x_0,\cdots,x_{n-1})$

*只要对单点集合成立, 对一般情况也成立, 证明见定理\ref{thm:independent_rv}

令
\[
\tilde{\PP}_n(\cdot):=\PN(\cdot|(X_0,\cdots,X_{n-1})=x_0^{n-1})=\PP(\cdot|(X_0,\cdots,X_{n})=x_0^{n})
\]
只证 $m=2$, 即由
\[\tilde{\PP}_n(X_{n+1}=x_{n+1})=\PP_{\{X_n=x_n\}}(X_{n+1}=x_{n+1})\] 
证得
\[\tilde{\PP}_n(X_{n+1}=x_{n+1}, X_{n+2}=x_{n+2})=\PP_{\{X_n=x_n\}}(X_{n+1}=x_{n+1}, X_{n+2}=x_{n+2})\]
\[
\begin{aligned}
    \tilde{\PP}_n((X_{n+1},X_{n+2})=(x_{n+1},x_{n+2}))&=\tilde{\PP}_n(X_{n+1}=x_{n+1})\cdot \tilde{\PP}_n(X_{n+2}=x_{n+2}|X_{n+1}=x_{n+1})\\
    &\overset{\eqref{eq:M1}}{=}\PP(X_{n+1}=x_{n+1}|X_n=x_n)\cdot \PP(X_{n+2}=x_{n+2}|X_{n+1}=x_{n+1})\\
    &\overset{\text{Cor}\eqref{cor:markov_con_cut}}{=}\PP(X_{n+1}=x_{n+1}|X_n=x_n)\cdot \PP(X_{n+2}=x_{n+2}|X_{n+1}=x_{n+1},X_n=x_n)\\
    &=\PN(X_{n+1}=x_{n+1})\cdot \PN(X_{n+2}=x_{n+2}|X_{n+1}=x_{n+1})\\
    &\overset{\eqref{eq:multiply_func}}{=}\PN((X_{n+1},X_{n+2})=(x_{n+1},x_{n+2}))
\end{aligned}
\]

\begin{corollary}
    设 $X$ 为马氏链, 则对每一个 $n\geq 1,m\geq 1, u_k<u_{k+1}, 0\leq k\leq n+m-1$, 有
    \[
    (X_{u_0},\cdots,X_{u_{n-1}})\ind_{\{X_{u_n}=x_{u_n}\}}(X_{u_{n+1}},\cdots,X_{u_{n+m}})
    \]
\end{corollary}

\newpage