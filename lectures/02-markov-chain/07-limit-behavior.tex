\subsection{极限行为与平稳分布的存在唯一性}
研究 $\displaystyle \lim_{n\to \infty}p_{ij}^{(n)}$
\begin{enumerate}
    \item 由 \eqref{eq:transient_multistep}, $j$暂留 $\displaystyle\Rightarrow \sum_{n\geq 0}p_{ij}^{(n)}<\infty,\forall i\in S\Rightarrow \lim_{n\to \infty}p_{ij}^{(n)}=0,\forall i\in S$. 下面可以把注意力放在常返上
    \item $\displaystyle\lim_{n\to \infty}p_{ij}^{(n)}$不存在的反例
    \[
    S=\{1,2\}, P=\begin{pmatrix}
        0&1\\
        1&0
    \end{pmatrix}, P^2=\begin{pmatrix}
        1&0\\
        0&1
    \end{pmatrix}
    \]
    \[
    P^{2n}=\begin{pmatrix}
        1&0\\
        0&1
    \end{pmatrix}, P^{2n+1}=P=\begin{pmatrix}
        0&1\\
        1&0
    \end{pmatrix}
    \]
    $p_{ij}^{(2n)}\neq p_{ij}^{(2n+1)},\forall i,j\in S$, 所以 $p_{ij}^{(n)}$ 不收敛
\end{enumerate}

\begin{definition}[周期]\label{def:cycle}
    令 $I_x:=\{n\geq 1|P_{xx}^{(n)}>0\}$, 定义 $x$ 的周期 $d(x)=\gcd(I_x)$
    \begin{enumerate}
        \item $d(x)>1$, 称 $x$ 周期的
        \item $d(x)=1$, 称 $x$ 非周期的
        \item $I_x=\emp$, 称 $x$ 周期为 $\infty$
    \end{enumerate}
    注: $\gcd$ 为 greatest common divisor 最大公因数.
\end{definition}

\begin{definition}
    称链是周期的, 若所有状态是周期的
\end{definition}

\begin{theorem}[收敛定理]
    马氏链不可约, 非周期, 且存在平稳分布 $\pi$, 则
    \[
    \lim_{n\to\infty}p_{ij}^{(n)}=\pi_j\quad (\forall i,j\in S)
    \]
    注:找到周期不是件容易的事, 我们通常讨论非周期的链
\end{theorem}

\begin{problem}[作用7-1]
    设 $S$ 有限, $\exists i\in S$, \stt $\lim_{n\to\infty}p_{ij}^{(n)}=\pi_j(\forall j\in S)$. 证明:$\pi=(\pi_j)_{j\in S}$是$P=(p_{ij})_{i,j\in S}$的平稳分布
\end{problem}

\begin{theorem}[渐进频率]\label{thm:asymptotic_frequency}
    马氏链不可约, 常返, 则
    \begin{equation}
\lim_{n\to\infty}\frac{N_n(y)}{n}=\frac{1}{\EE_yT_y}
\end{equation}
    注:\begin{enumerate}
        \item $N_n(y)=\sum_{k=1}^n\II_{\{X_k=y\}}$($n$时刻前, 访问$y$的总次数)
        \item 考虑 $\displaystyle\frac{N_n(y)}{n}$, 表示 $n$ 时刻前访问$y$的频率/时间比例, 因此 $\displaystyle\lim_{n\to\infty}\frac{N_n(y)}{n}$ 为在状态 $y$ 上花费的时间比例的极限
        \item $\displaystyle \EE_yT_y=\begin{cases}
            <\infty & $y\text{正常返}$\\
            \infty & $y\text{暂留/零常返}$
        \end{cases}$, rf. (Def \ref{def:recurrent_types}).
    \end{enumerate}
\end{theorem}

\begin{proof}
Durrett (3ed), Thm 1.20, p47.
\end{proof}

\begin{theorem}\label{thm:stationary_exists_unique}
    马氏链不可约
    \begin{enumerate}
        \item (平稳分布唯一性, Durrett, Thm 1.21) 若平稳分布存在, 则
        \begin{equation}
\pi_y=\frac{1}{\EE_yT_y}
\label{eq:thm1.21}
\end{equation}
        则$\pi$唯一
        \item (平稳测度存在性) 若马氏链常返, 则 $\exists$平稳测度, $\mu=(\mu_x)_{x\in S}$, 且 $\mu_x>0,\forall x$. 令 $T_x=\min\{n\geq 1|X_n=x\}$.
        \begin{equation}
\mu_x(y)=\sum_{n=0}^{\infty}\PP_x(X_n=y,T_x>n)
\label{eq:stationary_measure}
\end{equation}
    \end{enumerate}
    注:$\mu=(\mu_x)_{x\in S}$ 是一个平稳测度, 若
    \begin{enumerate}
        \item (测度) $\mu_x\geq 0,\forall x\in S$
        \item $\mu P=\mu$
    \end{enumerate}
\end{theorem}

\begin{proof}
\begin{enumerate}
\item \eqref{eq:thm1.21}: Durrett (3ed), Thm 1.21, p47.
\item \eqref{eq:stationary_measure}: Durrett (3ed), Thm 1.24, p48.
\end{enumerate}
\end{proof}

相对于上面的大定理, 下面的推论对我们更有用

\begin{corollary}
    马氏链具有有限状态, 不可约, 则
    \begin{enumerate}
        \item 存在唯一平稳分布 $\pi=(\pi_x)_{x\in S}$, 且 $\displaystyle\pi_x=\frac{1}{\EE_xT_x}>0,\forall x\in S$
        \item $\displaystyle\lim_{n\to\infty}\frac{N_n(y)}{n}=\frac{1}{\EE_xT_x}=\pi_x$
    \end{enumerate}
\end{corollary}

\begin{proof}
\begin{enumerate}
    \item $S$有限不可约, 闭集$\Rightarrow$ 不可约, 常返, rf. (Thm \ref{thm:finite-close-rec}).
    \begin{enumerate}
        \item 由Thm \ref{thm:stationary_exists_unique} (2)知, 存在$\mu=(\mu_x)_{x\in S},\mu_x\geq 0, \mu P=\mu$.令 $\displaystyle\pi_x=\frac{\mu_x}{\sum_{x\in S}\mu_x}$(正则化$\mu$), $\pi_x>0$, 且 $\displaystyle\pi P=\frac{1}{\sum_{x\in S}\mu_x}\mu P=\frac{1}{\sum_{x\in S}\mu_x}\mu=\pi$
        \item 由Thm \ref{thm:stationary_exists_unique} (1)知, $\pi$唯一且$\displaystyle\pi_x=\frac{1}{\EE_xT_x}$
    \end{enumerate}
    \item 由Thm \ref{thm:asymptotic_frequency}
\end{enumerate}
\end{proof}
\newpage
